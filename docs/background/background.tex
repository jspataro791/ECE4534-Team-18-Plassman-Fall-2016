\documentclass[12pt,letterpaper]{article}
\usepackage[utf8]{inputenc}
\usepackage{amsmath}
\usepackage{amsfonts}
\usepackage{amssymb}
\author{Andrew Bryant, Danny Dutton, Benjamin Singleton, John Spataro}
\title{Team 18 Background Research}
\begin{document}
\maketitle

\section{Network Communication}

\section{Line Following}
The SparkFun Line Following Array (LFA) is a device designed to provide information on the location of a line within the array. The sensor works by relaying information about the eight sensor "eyes", specifically which eye is currently sensing a contrasted line. 

The LFA is attached to the front of the rover. Its maximum detectable line width is 3/4 of an inch. The power source for the LFA is 5V at a maximum of 185mA. 

The protocol for the LFA is I2C. Example code is given in a Github repository provided with the LFA. Features of the LFA include a physically adjustable sensitivity potentiometer and toggleable sensors. 

Sunlight and shadowing can cause issues with LFA devices. Shadowing presents a false-contrast condition, while sunlight can overwhelm the sensor array with UV light, causing sporadic readings. In fact, it is useful to low-pass filter the results of the sensor in software to prevent such issues.

Another issue with LFA devices is the case where no line is detected at all. In this case, information about the last known line position needs to be taken into account, or the LFA will cause a rover to lose its location. This is usually done by keeping a record of some previous sensor states, and when the "no detection" state occurs the record must be accessed to determine the best route to relocation.



\section{Path Finding}

\section{Pixycam}


\end{document}